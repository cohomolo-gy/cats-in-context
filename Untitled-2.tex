\documentclass[11pt, oneside]{article}   	% use "amsart" instead of "article" for AMSLaTeX format
\usepackage{geometry}                		% See geometry.pdf to learn the layout options. There are lots.
\geometry{letterpaper}                   		% ... or a4paper or a5paper or ... 
\usepackage[parfill]{parskip}    		% Activate to begin paragraphs with an empty line rather than an indent
\usepackage{graphicx}				% Use pdf, png, jpg, or eps§ with pdflatex; use eps in DVI mode
								% TeX will automatically convert eps --> pdf in pdflatex		
\usepackage{amssymb}
\usepackage{hyperref}



\title{NYC Category Theory Meetup}
\author{Emily Pillmore}
%\date{}							% Activate to display a given date or no date

\begin{document}
\maketitle
\section{Syllabus}

This meetup will be full dive into Category Theory, aimed at the sophisticated grade student, or a particularly dedicated undergrad.

\subsection{Logistics}

The meetup will be held Sundays at 5pm, at the Kadena Office in Gowanus - 92 3rd St. Brooklyn, NY, 11231 (just ring the bell for the Kadena office). You can find us on the 2nd floor, a block and a half off of the Carroll St. F stop. Pizza and drinks will be provided on location if I can schedule them in time.

\subsection{Textbook}

We'll be using the book \textit{Category Theory in Context} by Emily Riehl. You can find this book for \$30 on Amazon, paperback. Alternatively, if you cannot afford the book, there is a free version here: \href{http://www.math.jhu.edu/eriehl/context.pdf}{Category Theory in Context}

Additionally, we may refer to \textit{Categories for the Working Mathematician} by Saunders MacLane, and \textit{Seven Sketches in Compositionality} by Fong and Spivak. The latter two are nice-to-haves, but not necessary. Much of the content for the group can be cobbled together from various blog posts and alternative resources.

\subsection{Misc}

Meetup materials will be hosted at \href{https://github.com/cohomolo-gy/cats-in-context}{cohomolo-gy}, and discussion will be held in \#categorytheory in the following discord: \href{Haskell $\cap$ Chat}{https://discord.gg/DHQgRq2}

\end{document}  