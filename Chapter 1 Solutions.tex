\documentclass[10pt, oneside]{article}   	% use "amsart" instead of "article" for AMSLaTeX format
\usepackage{geometry}                		% See geometry.pdf to learn the layout options. There are lots.
\geometry{letterpaper}                   		% ... or a4paper or a5paper or ... 
\usepackage[parfill]{parskip}    		% Activate to begin paragraphs with an empty line rather than an indent
\usepackage{graphicx}				% Use pdf, png, jpg, or eps§ with pdflatex; use eps in DVI mode
								% TeX will automatically convert eps --> pdf in pdflatex		
\usepackage{amssymb}
\usepackage{amsfonts}
\usepackage{amsmath}
\usepackage{bm}
\usepackage{tikz-cd}
\usepackage{enumerate}
\usepackage{xfrac}
\usepackage{hyperref}
\usepackage{unicode-math}

\newcommand{\cat}[1]{\bm{ \mathsf{#1} }}
\newcommand{\functor}[3]{#1 : \cat{#2} \to \cat{#3}}
\newcommand{\functordef}{\functor{F}{C}{D}}
\newcommand{\cc}{\cat{C}}
\newcommand{\dd}{\cat{D}}
\newcommand{\ee}{\cat{E}}
\newcommand{\subcat}[2]{\bm{ \mathsf{#1}}_{\bm{ \mathsf{#2}}}}
\newcommand{\op}[1]{#1^{\text{op}}}
\newcommand{\opc}{\op{\cc}}
\newcommand{\opd}{\op{\dd}}
\newcommand{\ope}{\op{\ee}}
\newcommand{\mono}{\rightarrowtail}
\newcommand{\epi}{\twoheadrightarrow}
\newcommand{\one}{\mathbb{1}}
%SetFonts

%SetFonts


\title{Categories in Context - Ch. 1 Solutions}

\author{NYC Categories Theory Meetup}

\begin{document}
\maketitle

\section{Categories, Functors, Natural Transformations}

This chapter describes the fundamental data associated with categorical structure and its structure-preserving maps called \textbf{Functors}. Additionally, we learn about the maps between functors, \textbf{Natural Transformations}, and their coherence data. Other topics include sizing constraints, additional structure of morphisms within a category, and begin to understand arguments from duality and via the art of the diagram chase.

\subsection{Abstract + Concrete Categories}

In this section, we discuss the data associated with categories. 

\subsubsection{}

\begin{enumerate}[i.]
\item 

\begin{enumerate}[(i)] 
\item \textit{Show that a morphism can have at most one inverse isomorphism}. 

We proceed by routine proof of uniqueness. Let $f : x \to y$ be a morphism with $g, h : y \rightrightarrows x$ be an inverse isomorphisms for $f$. By definition, if $g$ or $h$ are isomorphisms, then $fg = fh = 1_y$ and $gf = hf = 1_x$. Thus we have that 
$g(fh) = g(fg) = g1_y = g$ and $(gf)g = (gf)h = 1_xh = h$ (parentheses added for emphasis). Hence, $g = h$.

\item \textit{Consider a morphism $f : x \to y$. Show that if there exists a pair of morphisms $g, h: y \rightrightarrows x$ so that $gf = 1_x$ and $fh = 1_y$, then $g = h$ and $f$ is an isomorphism.}

This shows that we can weaken the above to understand that for a given $f: x \to y$, \textit{any} pair of parallel morphisms $g,h: y \rightrightarrows x$ such that if $gf = 1_x$ and $fh = 1_y$ implies $g = h$ and $f$ is an isomorphism. The notion being described here is that an isomorphism is "built", so to speak, by the data of a right and left inverse. If these inverses coincide, then one has an isomorphism! 

As before, note that if $gf = 1_x$, then $(gf)h = 1_x h = h$. Likewise, we have that $fh = 1_y$, and therefore $g(fh) = g1_y = g$. Hence, $fgfh = fg = fh$, which implies $fg = 1_y$. Hence, $g$ is a two-sided inverse, and $h$ is as well, and hence, are isomorphisms. According to the previous exercise, these must coincide.

\item \textit{Let $\cc$ be a category. Show that the collection of isomorphisms in $\cc$ defines a subcategory, the \textbf{maximal groupoid} of $\cc$.} 

First, let's show that the collection of isomorphisms of $\cc$ defines a subcategory, $\subcat{Iso}{C}$ which has a groupoid structure. Then, we will prove it is maximal in a precise sense. 

Note that for any object $c \in \cc$, we require that $c \in  \subcat{Iso}{C}$, since the identity morphism is an isomorphism. Thus the objects of $\subcat{Iso}{C}$ must be those of $\cc$. Additionally, composition in $\subcat{Iso}{C}$ is that of $\cc$. Indeed, it is a groupoid by definition, since all isomorphisms are invertible, and this subcategory $\subcat{Iso}{C}$.

Next, we must show that it is maximal. To make this precise, we simply show that for any other groupoid $\cat{G}$, then $\cat{G}$ is a subcategory of $\subcat{Iso}{C}$, or is  $\subcat{Iso}{C}$ itself. Let $\cat{G}$ be another maximal groupoid. Since $\cat{G}$ is the collection of all isomorphisms for all objects in $\cc$, we must have that every object and every isomorphism of $\subcat{Iso}{C}$ is in $\cat{G}$, hence, $\subcat{Iso}{C}$ is a subcategory of $\cat{G}$. Likewise, $\cat{G}$ is a subcategory of $\subcat{Iso}{C}$.There are no differences between these categories because isomorphisms are unique, and since they define the same objects and morphisms, they are identical. 
\end{enumerate}


\item For any category, show that: 
\begin{enumerate}[(i)]

\item There is a category $c / \cc$, whose objects are morphisms $f: c \to x$ with domain $c$ and in which a morphisms from $f : c \to x$ to $g: c \to y$ is a map $h : x \to y$ between the codomains so that the triangle

\begin{center}
\begin{tikzcd}
& c \ar[dl, "f", swap] \ar[dr, "g"]
\\ x \ar[rr, "h", swap] & & y
\end{tikzcd}
\end{center}
commutes, i.e., so that $g = hf$. 

\newpage
We must show that the above forms a category $c / \cc$ with objects that are morphisms $f : c \to x$, $g : c \to y$ in $\cc$, and morphisms $h : x \to y$. Consider the following triangles, which are the identity for $f$ and composite arrow for $h'h$ for $h': y \to z$, respectively:

\[
\begin{aligned}
\begin{tikzcd} 
& c \ar[dl, "f", swap] \ar[dr, "g"]
\\ x \ar[rr, "1_x", swap] & & x
\end{tikzcd}
\qquad
\begin{tikzcd} 
& c \ar[dl, "f", swap] \ar[dr, "g"] \ar[rr, equal] & & c \ar[dl, "g", swap] \ar[dr, "k"]
\\ x \ar[rr, "h", swap] \ar[rrrr, "h'h", bend right] & & y \ar[rr, "h'", swap] & & z
\end{tikzcd}
\end{aligned}
\]

Note that composition of the latter triangles is associative, inheriting associativity from the underlying category $\cc$, and the identity triangle is indeed an identity as a result. Hence, $c / \cc$ is a category, and dually, $\cc / c$ is a category, satisfying $(ii)$. 
\end{enumerate}
\end{enumerate}

\subsection{Duality}

In this section, we discuss additional structure of morphisms, and introduce the notion of "duality" and opposite categories $\cc^{op}$. 

\subsubsection{}

\begin{enumerate}[(i)]
\item Show that $\cc/c \cong \op{(c /\opc)}$. Defining $\cc / c$ to be $\op{(c /\opc)}$, deduce the last 2 exercises from section 1.1.

Note that $c/\opc$ is simply $c / \cc$ with the morphisms reversed. Morphisms in the original category are simply arrows $h : x \to y$ between the codomains of objects $f :  c \to x$ and $g : c \to y$, hence, morphisms in $c / \opc$ simply reverse direction, so that $h$ becomes $\op{h}: y \to x$. Indeed, if we consider $\op{(c / \opc)}$, then the opposite functor takes $f,g$ to $\op{f}, \op{g}$ and $\op{h}$ to $\op{(\op{h})} = h$. Thus we can see that the direction of $f$ and $g$ reverse, but $h$ is as it was defined - a morphism in $\cc$ again. However, the direction of $f$ and $g$ has reversed, we now have objects $\op{f} : x \to c$, and $\op{g} : y \to c$. This is precisely the data of $\cc / c$. 
\end{enumerate}

\subsubsection{}
\begin{enumerate}[(i)]

\item Show that a morphism $f: x \to y$ is a split epimorphism in a category $\cc$ if and only if for all $c \in \cc$, post-composition $f_*: \cc(c,x) \to \cc(c,y)$ defines a surjective function. 

Lets show the original implication $(\Rightarrow)$ first. Let $f : x \to y$ be a split epimorphism. By definition, a morphism is a split-epimorphism if there exists a monomorphism $s : y \to x$ which is a right inverse of $f$ - i.e. $fs = 1_y$.  Note that since $s$ is monic, by definition 1.2.7, post-composition by $s$ defines an injection $s_* : \cc(c,y) \to \cc(c, x)$. Now consider post-composition by $f$. $f_*s_* = 1_{\cc(c,x)}$, hence the induced $f_*$ and $s_*$ is also a split monic and epic pair, and hence, $f_*$ is a surjection. 

Now lets show $(\Rightarrow)$. Let $f_*$ be a surjection for all $c \in \cc$. Note that in $\opc$, $\op{f_*} : \opc(y,c) \to \opc(x,c)$ is an injection. By 1.2.7, this defines an epimorphism in $f: x \to y$ in $\cc$.

\item Argue by duality... 

Left up to meetup. This one isn't bad, just tedious.

\end{enumerate}

\subsubsection{}

Lets do $(i)$ and $(ii')$ for flavor, but we should probably do the others in the meetup.

\begin{enumerate}[(i)]

\item If $f : x \mono y$ is monic and $g : y \mono$ is monic, then $gf : x \mono z$ is also monic.

Let $h,k : w \rightrightarrows x$ and suppose $gfh = gfk$. We will show that $gf$ is monic. Note that by associativity, $g(fh) = g(fk)$ implies that $fh = fk$ since $g$ is mono. Likewise, $fh = fk$ implies $h = k$ since $f$ is mono. Hence, $(gf)h = (gf)k$ implies $h = k$. Thus, $gf$ is monic. 

\item If $f : x \to y$ and $g : y \to z$ are morphisms, then if $gf$ is epic, $g$ is epic. 

Suppose $gf$ is epic. Let $h, k : z \to w$, and suppose $hgf = kgf$. Since $gf$ is epic, this implies $h = k$. However, note that $hg = kg$, then $hgf = kgf$ implies $h = k$ by epicness of $gf$. Hence, $g$ must be epic.

Now, note that monomorphisms (resp. epimorphisms) are closed under composition and defines a trivial monomorphism (resp. epimorphism) for all for every $c \in \cc$. Thus, the class of mono- and epimorphisms define subcategories of $\cc$.
\end{enumerate}

\subsubsection{}

Jumping to exercise $1.2.iv$. 

\begin{enumerate}[(i)]

\item Prove that a morphism that is both a monomorphism and a split epimorphism is necessarily an isomorphism. Prove its dual statement. 

Let $f: x \to y$ be a monomorphism and a retract for $s : y \to x$. To prove $f$ is an isomorphism, we must prove that $f$ has a two-sided inverse. Note that $s$ is probably a good candidate, since it is already a one-sided inverse for $f$ - $fs = 1_y$. Note that $s$ is the unique right inverse of $f$, since if there were another $s' : y \to x$, then if $fs = fs'$, then $s = s'$ because $f$ is monic. Note also, by $f$'s monic properties that $fsf = f1_x$ implies $sf = 1_x$. Hence $s$ is a unique two-sided inverse for $f$. 

Flip the arrows for the dual statement. Lets do this in the meetup. 
\end{enumerate}

\subsection{Functoriality}


\begin{enumerate}[(i)]

\item What is a functor between groups, regarded as one-object categories?

The category structure of a group seen as a one-object category $\cat{BG}$ contains the following data: there is a single object, $*$, and morphisms the whole of the group $G$. Thus, each element of $G$ can be seen as an element collection of endomorphisms $\cat{BG}(*,*)$. A functor of groups $F: \cat{BG} \to cat{BG'}$ takes objects to objects and morphisms to morphisms. Hence, $F : * \mapsto *$ and $F(gh) = FgFh = g'h'$ in $\cat{BG'}$. Note also that the identity element in $G$ maps to the $1_*$, hence, $F1_* = 1_{F* = *}$, which tells us that identities map to identities. This is precisely the structure of the group homomorphism with which we are familiar.

\item What is a functor between preorders regarded as categories? 

Monotonic functions. 

\item Verify that the constructions introduced in 1.3.11 are functorial 

Lets do this in meetup

\item Find an example to show that the objects and morphisms in the image of a functor $F : \cc \to \dd$ do not necessarily define a subcategory of $\dd$. 

\href{This MO question}{https://math.stackexchange.com/questions/413138/can-it-happen-that-the-image-of-a-functor-is-not-a-category}


\item What is the difference between a functor $\opc \to \dd$ and a functor $\cc \to \opd$? What is the difference between a functor $\cc \to \dd$ and $\opc \to \opd$?

A functor $F : \opc \to \dd$ maps arrows $d \to c$ to arrows $Fc \to Fd$, while $F : \cc \to \opd$ maps $c \to d$ to $Fd \to Fc$. The outcome? roughly the same. It depends on the intent. Note that the two constructions are dual, but $F$ is always a contravariant funtor. Likewise, next case, $F$ is always a covariant functor. There is a kind of polarity thing going on here, as well, where one can view variance as positive or negative polarity. The composition of two co- or contravariant functors is a covariant functor, and the composition of contra with covariant is contravariant no matter what. 

\item Given functors $ F : \dd \to \cc$ and $G : \ee \to \cc$, show that there is a category, called the \textbf{comma category} $F \downarrow G$ which has 

\begin{itemize}
\item as objects, triples $(d \in \dd, e \in \ee, f : Fd \to Ge \in \cc)$, and 
\item as morphisms $(d, e, f) \to (d', e', f')$, a pair of morphisms $(h : d \to d', k: e \to e')$ so that the square 

\begin{center}
\begin{tikzcd}
Fd \ar[r, "f"] \ar[d, "Fh", swap] & Ge \ar[d, "Gk"]
\\ Fd' \ar[r, "f'", swap] & Ge'
\end{tikzcd}
\end{center}

commutes in $\cc$, i.e., so that $f' \cdot Fh = Gk \cdot f$. 
\end{itemize}

Define a pair of projection functors $dom : F \downarrow G \to \dd$ and $cod : F \downarrow G \to \ee$. 


First, note the target categories are $\dd$ and $\ee$ respectively for these projection functors. Define $dom$ and $cod$ as follows: For each object (triple) in $F \downarrow G$, take first projections to be $dom$, and for $cod$, second projections. 

\item Define functors to construct the slice categories $c / \cc$ and $\cc / c$ as special cases of comma categories. What are the projection functors?

Let $F : \one \to \cc$ be the functor from the terminal category $\one$ to the object $c \in \cc$. Let $G \cong 1_{\cc}$. Then, the data of the comma category construction gives us objects $(* \in \one, x \in \cc, f : c \to x)$, and morphisms $(*, x, f: c \to x) \to (*, y, g: c \to y)$ given by a pair $(1_*, k: x \to y)$. 

\newpage

Thus, we have the square

\begin{center}
\begin{tikzcd}
c \ar[r, "f"] \ar[d, equal] & x \ar[d, "h"]
\\ c \ar[r, "g", swap] & y
\end{tikzcd}
\end{center}

Dually, we, swap $F$ and $G$ to achieve the over slice category. The projection functors for this data  remain the same. 

\item Lets do this one in the Meetup.
\end{enumerate}

\subsection{Naturality}

This section begins to discuss naturality, natural transformations, and (de)categorification. 

\begin{enumerate}[(i)]
\item 
\end{enumerate}
\end{document}  