\documentclass[10pt, oneside]{article}   	% use "amsart" instead of "article" for AMSLaTeX format
\usepackage{geometry}                		% See geometry.pdf to learn the layout options. There are lots.
\geometry{letterpaper}                   		% ... or a4paper or a5paper or ... 
\usepackage[parfill]{parskip}    		% Activate to begin paragraphs with an empty line rather than an indent
\usepackage{graphicx}				% Use pdf, png, jpg, or eps§ with pdflatex; use eps in DVI mode
								% TeX will automatically convert eps --> pdf in pdflatex		
\usepackage{amssymb}
\usepackage{amsfonts}
\usepackage{amsmath}
\usepackage{bm}
\usepackage{tikz-cd}
\usepackage{enumerate}
\usepackage{xfrac}
\usepackage{hyperref}
\usepackage{unicode-math}

\newcommand{\cat}[1]{\bm{ \mathsf{#1} }}
\newcommand{\functor}[3]{#1 : \cat{#2} \to \cat{#3}}
\newcommand{\functordef}{\functor{F}{C}{D}}
\newcommand{\cc}{\cat{C}}
\newcommand{\dd}{\cat{D}}
\newcommand{\ee}{\cat{E}}
\newcommand{\subcat}[2]{\bm{ \mathsf{#1}}_{\bm{ \mathsf{#2}}}}
\newcommand{\op}[1]{#1^{\text{op}}}
\newcommand{\opc}{\op{\cc}}
\newcommand{\opd}{\op{\dd}}
\newcommand{\ope}{\op{\ee}}
\newcommand{\mono}{\rightarrowtail}
\newcommand{\epi}{\twoheadrightarrow}
\newcommand{\bg}{\cat{BG}}
\newcommand{\bgg}{\cat{BG'}}
\newcommand{\nt}{\Rightarrow}
\newcommand{\ant}[2]{\alpha : F \nt G} 
\newcommand{\bnt}[2]{\beta : F \nt G} 
\newcommand{\anti}[2]{\alpha : F \cong G} 
\newcommand{\bnti}[2]{\beta : F \cong G} 
\newcommand{\zero}{\mathbb{0}}
\newcommand{\one}{\mathbb{1}}
\newcommand{\two}{\mathbb{2}}
\newcommand{\three}{\mathbb{3}}
%SetFonts

%SetFonts


\title{Categories in Context - Ch. 1 Solutions}

\author{NYC Categories Theory Meetup}

\begin{document}
\maketitle

\section{Categories, Functors, Natural Transformations}

This chapter describes the fundamental data associated with categorical structure and its structure-preserving maps called \textbf{Functors}. Additionally, we learn about the maps between functors, \textbf{Natural Transformations}, and their coherence data. Other topics include sizing constraints, additional structure of morphisms within a category, and begin to understand arguments from duality and via the art of the diagram chase.

\subsection{Abstract + Concrete Categories}

In this section, we discuss the data associated with categories. 

\subsubsection{}

\begin{enumerate}[i.]
\item 

\begin{enumerate}[(i)] 
\item \textit{Show that a morphism can have at most one inverse isomorphism}. 

We proceed by routine proof of uniqueness. Let $f : x \to y$ be a morphism with $g, h : y \rightrightarrows x$ be inverse isomorphisms for $f$. By definition, if $g$ or $h$ are isomorphisms, then $fg = fh = 1_y$ and $gf = hf = 1_x$. Thus we have that 
$g(fh) = g(fg) = g1_y = g$ and $(gf)g = (gf)h = 1_xh = h$ (parentheses added for emphasis). Hence, $g = h$.

\item \textit{Consider a morphism $f : x \to y$. Show that if there exists a pair of morphisms $g, h: y \rightrightarrows x$ so that $gf = 1_x$ and $fh = 1_y$, then $g = h$ and $f$ is an isomorphism.}

This shows that we can weaken the above to understand that for a given $f: x \to y$, \textit{any} pair of parallel morphisms $g,h: y \rightrightarrows x$ such that $gf = 1_x$ and $fh = 1_y$ implies $g = h$ and $f$ is an isomorphism. The notion being described here is that an isomorphism is "built", so to speak, by the data of a right and left inverse. If these inverses coincide, then one has an isomorphism! 

As before, note that if $gf = 1_x$, then $(gf)h = 1_x h = h$. Likewise, we have that $fh = 1_y$, and therefore $g(fh) = g1_y = g$. Hence, $fgfh = fg = fh$, which implies $fg = 1_y$. Hence, $g$ is a two-sided inverse, and $h$ is as well, and hence, are isomorphisms. According to the previous exercise, these must coincide.

\item \textit{Let $\cc$ be a category. Show that the collection of isomorphisms in $\cc$ defines a subcategory, the \textbf{maximal groupoid} of $\cc$.} 

First, let's show that the collection of isomorphisms of $\cc$ defines a subcategory, $\subcat{Iso}{C}$ which has a groupoid structure. Then, we will prove it is maximal in a precise sense. 

Note that for any object $c \in \cc$, we require that $c \in  \subcat{Iso}{C}$, since the identity morphism is an isomorphism. Thus the objects of $\subcat{Iso}{C}$ must be those of $\cc$. Additionally, composition in $\subcat{Iso}{C}$ is that of $\cc$. Indeed, it is a groupoid by definition, since all isomorphisms are invertible, and this subcategory $\subcat{Iso}{C}$.

Next, we must show that it is maximal. To make this precise, we simply show that for any other groupoid $\cat{G}$, then $\cat{G}$ is a subcategory of $\subcat{Iso}{C}$, or is  $\subcat{Iso}{C}$ itself. Let $\cat{G}$ be another maximal groupoid. Since $\cat{G}$ is the collection of all isomorphisms for all objects in $\cc$, we must have that every object and every isomorphism of $\subcat{Iso}{C}$ is in $\cat{G}$, hence, $\subcat{Iso}{C}$ is a subcategory of $\cat{G}$. Likewise, $\cat{G}$ is a subcategory of $\subcat{Iso}{C}$.There are no differences between these categories because isomorphisms are unique, and since they define the same objects and morphisms, they are identical. 
\end{enumerate}


\item For any category, show that: 
\begin{enumerate}[(i)]

\item There is a category $c / \cc$, whose objects are morphisms $f: c \to x$ with domain $c$ and in which a morphisms from $f : c \to x$ to $g: c \to y$ is a map $h : x \to y$ between the codomains so that the triangle

\begin{center}
\begin{tikzcd}
& c \ar[dl, "f", swap] \ar[dr, "g"]
\\ x \ar[rr, "h", swap] & & y
\end{tikzcd}
\end{center}
commutes, i.e., so that $g = hf$. 

We must show that the above forms a category $c / \cc$ with objects that are morphisms $f : c \to x$, $g : c \to y$ in $\cc$, and morphisms $h : x \to y$. Consider the following triangles, which are the identity for $f$ and composite arrow for $h'h$ for $h': y \to z$, respectively:

\[
\begin{aligned}
\begin{tikzcd} 
& c \ar[dl, "f", swap] \ar[dr, "f"]
\\ x \ar[rr, "1_x", swap] & & x
\end{tikzcd}
\qquad
\begin{tikzcd} 
& c \ar[dl, "f", swap] \ar[dr, "g"] \ar[rr, equal] & & c \ar[dl, "g", swap] \ar[dr, "k"]
\\ x \ar[rr, "h", swap] \ar[rrrr, "h'h", bend right] & & y \ar[rr, "h'", swap] & & z
\end{tikzcd}
\end{aligned}
\]

Note that composition of the latter triangles is associative, inheriting associativity from the underlying category $\cc$, and the identity triangle is indeed an identity as a result. Hence, $c / \cc$ is a category, and dually, $\cc / c$ is a category, satisfying $(ii)$. 
\end{enumerate}
\end{enumerate}

\subsection{Duality}

In this section, we discuss additional structure of morphisms, and introduce the notion of "duality" and opposite categories $\cc^{op}$. 

\subsubsection{}

\begin{enumerate}[(i)]
\item Show that $\cc/c \cong \op{(c /\opc)}$. Defining $\cc / c$ to be $\op{(c /\opc)}$, deduce the last 2 exercises from section 1.1.

Note that $c/\opc$ is simply $c / \cc$ with the morphisms reversed. Morphisms in the original category are simply arrows $h : x \to y$ between the codomains of objects $f :  c \to x$ and $g : c \to y$, hence, morphisms in $c / \opc$ simply reverse direction, so that $h$ becomes $\op{h}: y \to x$. Indeed, if we consider $\op{(c / \opc)}$, then the opposite functor takes $f,g$ to $\op{f}, \op{g}$ and $\op{h}$ to $\op{(\op{h})} = h$. Thus we can see that the direction of $f$ and $g$ reverse, but $h$ is as it was defined - a morphism in $\cc$ again. However, the direction of $f$ and $g$ has reversed, we now have objects $\op{f} : x \to c$, and $\op{g} : y \to c$. This is precisely the data of $\cc / c$. 
\end{enumerate}

\subsubsection{}
\begin{enumerate}[(i)]

\item Show that a morphism $f: x \to y$ is a split epimorphism in a category $\cc$ if and only if for all $c \in \cc$, post-composition $f_*: \cc(c,x) \to \cc(c,y)$ defines a surjective function. 

Lets show the original implication $(\Rightarrow)$ first. Let $f : x \to y$ be a split epimorphism. By definition, a morphism is a split-epimorphism if there exists a monomorphism $s : y \to x$ which is a right inverse of $f$ - i.e. $fs = 1_y$.  Note that since $s$ is monic, by definition 1.2.7, post-composition by $s$ defines an injection $s_* : \cc(c,y) \to \cc(c, x)$. Now consider post-composition by $f$. $f_*s_* = 1_{\cc(c,x)}$, hence the induced $f_*$ and $s_*$ is also a split monic and epic pair, and hence, $f_*$ is a surjection. 

Alternatively, suppose $s$ and $f$ are as above. Let $g : c \to y$ be a morphism. Consider $t = sg : c \to x$. Then, $g = 1_yg = (fs)g = f(sg) = t$, whence, $fg = t$. This ranges over all $c, x, y$, hence, $f_*$ is a surjection. 

Now lets show $(\Leftarrow)$. Since $f_*$ is surjective on all points $c \in \cc$,  let $c = y$. Thus, we have a function we have a function $1_y \in \cc(y,y)$ and a surjective function $f_*$ : \cc(y,x) \to \cc(y,y), such that there exists some $g$ with $fg = 1_y$. Hence, $f$ splits. 

\item Argue by duality... 

Left up to meetup. This one isn't bad, just tedious.

\end{enumerate}

\subsubsection{}

Lets do $(i)$ and $(ii')$ for flavor, but we should probably do the others in the meetup.

\begin{enumerate}[(i)]

\item If $f : x \mono y$ is monic and $g : y \mono$ is monic, then $gf : x \mono z$ is also monic.

Let $h,k : w \rightrightarrows x$ and suppose $gfh = gfk$. We will show that $gf$ is monic. Note that by associativity, $g(fh) = g(fk)$ implies that $fh = fk$ since $g$ is mono. Likewise, $fh = fk$ implies $h = k$ since $f$ is mono. Hence, $(gf)h = (gf)k$ implies $h = k$. Thus, $gf$ is monic. 

\item If $f : x \to y$ and $g : y \to z$ are morphisms, then if $gf$ is epic, $g$ is epic. 

Suppose $gf$ is epic. Let $h, k : z \to w$, and suppose $hgf = kgf$. Since $gf$ is epic, this implies $h = k$. However, note that if $hg = kg$, then $hgf = kgf$ implies $h = k$ by epicness of $gf$. Hence, $g$ must be epic.

Now, note that monomorphisms (resp. epimorphisms) are closed under composition and the identity for defines a trivial monomorphism (resp. epimorphism) for all for every $c \in \cc$. Thus, the class of mono- and epimorphisms define subcategories of $\cc$.
\end{enumerate}

\subsubsection{}

Jumping to exercise $1.2.iv$. 

\begin{enumerate}[(i)]

\item Prove that a morphism that is both a monomorphism and a split epimorphism is necessarily an isomorphism. Prove its dual statement. 

Let $f: x \to y$ be a monomorphism and a retract for $s : y \to x$. To prove $f$ is an isomorphism, we must prove that $f$ has a two-sided inverse. Note that $s$ is probably a good candidate, since it is already a one-sided inverse for $f$ - $fs = 1_y$. Note that $s$ is the unique right inverse of $f$, since if there were another $s' : y \to x$, then if $fs = fs'$, then $s = s'$ because $f$ is monic. Note also, by $f$'s monic properties that $fsf = f1_x$ implies $sf = 1_x$. Hence $s$ is a unique two-sided inverse for $f$. 

Flip the arrows for the dual statement. Lets do this in the meetup. 
\end{enumerate}

\subsection{Functoriality}


\begin{enumerate}[(i)]

\item What is a functor between groups, regarded as one-object categories?

The category structure of a group seen as a one-object category $\cat{BG}$ contains the following data: there is a single object, $*$, and morphisms the whole of the group $G$. Thus, each element of $G$ can be seen as an element collection of endomorphisms $\cat{BG}(*,*)$. A functor of groups $F: \cat{BG} \to \cat{BG'}$ takes objects to objects and morphisms to morphisms. Hence, $F : * \mapsto *$ and $F(gh) = FgFh = g'h'$ in $\cat{BG'}$. Note also that the identity element in $G$ maps to the $1_*$, hence, $F1_* = 1_{F* = *}$, which tells us that identities map to identities. This is precisely the structure of the group homomorphism with which we are familiar.

\item What is a functor between preorders regarded as categories? 

Monotonic functions. 

\item Verify that the constructions introduced in 1.3.11 are functorial 

Lets do this in meetup

\item Find an example to show that the objects and morphisms in the image of a functor $F : \cc \to \dd$ do not necessarily define a subcategory of $\dd$. 

\href{This MO question}{https://math.stackexchange.com/questions/413138/can-it-happen-that-the-image-of-a-functor-is-not-a-category}


\item What is the difference between a functor $\opc \to \dd$ and a functor $\cc \to \opd$? What is the difference between a functor $\cc \to \dd$ and $\opc \to \opd$?

A functor $F : \opc \to \dd$ maps arrows $d \to c$ to arrows $Fc \to Fd$, while $F : \cc \to \opd$ maps $c \to d$ to $Fd \to Fc$. The outcome? roughly the same. It depends on the intent. Note that the two constructions are dual, but $F$ is always a contravariant funtor. Likewise, next case, $F$ is always a covariant functor. There is a kind of polarity thing going on here, as well, where one can view variance as positive or negative polarity. The composition of two co- or contravariant functors is a covariant functor, and the composition of contra with covariant is contravariant no matter what. 

\item Given functors $ F : \dd \to \cc$ and $G : \ee \to \cc$, show that there is a category, called the \textbf{comma category} $F \downarrow G$ which has 

\begin{itemize}
\item as objects, triples $(d \in \dd, e \in \ee, f : Fd \to Ge \in \cc)$, and 
\item as morphisms $(d, e, f) \to (d', e', f')$, a pair of morphisms $(h : d \to d', k: e \to e')$ so that the square 

\begin{center}
\begin{tikzcd}
Fd \ar[r, "f"] \ar[d, "h", swap] & Ge \ar[d, "k"]
\\ Fd' \ar[r, "f'", swap] & Ge'
\end{tikzcd}
\end{center}

commutes in $\cc$, i.e., so that $f' \cdot Fh = Gk \cdot f$. 
\end{itemize}

Define a pair of projection functors $dom : F \downarrow G \to \dd$ and $cod : F \downarrow G \to \ee$. 


To prove that $F \downarrow G$ is a category, we must show that composition is well-defined, and that every object has an associated identity morphism. The latter is easiest: the identity morphism for a triple $(d, e, f)$ is simply the data $1_{(d,e,f)} = (1_d, 1_e)$. It's clear that this is an identity, since, for any other $f = (h : d \to d', k: e \to e')$, and $f1_{(d,e,f)} = 1_{(d',e',f')}f = f$. Also note that for any other morphism $g = (h': d' \to d'', k' : e' \to e'')$, then composition is that of commutative squares: 

\begin{center}
\begin{tikzcd}
Fd \ar[r, "f"] \ar[d, "h", swap] & Ge \ar[d, "k"]
\\ Fd' \ar[r, "f'", swap] \ar[d, "h'", swap] & Ge' \ar[d, "k"]
\\ Fd'' \ar[r, "f''", swap] & Ge''

\end{tikzcd}
\end{center}

Note that since commutative squares are, well, commutative, that "pasting" squares together along one side or another is well-defined. Now, lets define some projections. 

First, note the target categories are $\dd$ and $\ee$ respectively for these projection functors. Define $dom$ and $cod$ as follows: For each object (triple) in $F \downarrow G$, take first projections to be $dom$, and for $cod$, second projections. 

\item Define functors to construct the slice categories $c / \cc$ and $\cc / c$ as special cases of comma categories. What are the projection functors?

Let $F : \one \to \cc$ be the functor from the terminal category $\one$ to the object $c \in \cc$. Let $G \cong 1_{\cc}$. Then, the data of the comma category construction gives us objects $(* \in \one, x \in \cc, f : c \to x)$, and morphisms $(*, x, f: c \to x) \to (*, y, g: c \to y)$ given by a pair $(1_*, k: x \to y)$. 

Thus, we have the square

\begin{center}
\begin{tikzcd}
c \ar[r, "f"] \ar[d, equal] & x \ar[d, "h"]
\\ c \ar[r, "g", swap] & y
\end{tikzcd}
\end{center}

Dually, we, swap $F$ and $G$ to achieve the over slice category. The projection functors for this data  remain the same. 

\item Lets do this one in the Meetup.
\end{enumerate}

\subsection{Naturality}

This section begins to discuss naturality, natural transformations, and (de)categorification. 

\begin{enumerate}[(i)]
\item Suppose $\ant{F}{G}$ is a natural isomorphism. Show that the inverses of the component morphisms define the components of a natural isomorphism $\alpha^{-1} : G \Rightarrow F$

Let $\alpha$ be a natural isomorphism. By definition, this means that every component $\alpha_c$ is an isomorphism. Hence, every component $\alpha_c$ has a unique two-sided inverse - call it $\alpha^{-1}_c : Gc \to Fc$. Since $\alpha_c$ ranges over all $c \in \cc$, so too must $\alpha^{-1}_c$, defining precisely the data of a transformation $G \Rightarrow F$. Naturality follows from the naturality of $\alpha_c$. For if each component $\alpha^{-1}_c$ is an inverse for $\alpha_c$, then the following rectangle must commute for some $f : c \to d$: 

\begin{center}
\begin{tikzcd}

Gc \ar[r, "\alpha^{-1}_c"] \ar[d, "Gf", swap] \ar[rr, "1_{Gc}", bend left = 40] & Fc \ar[d, "Ff"] \ar[r, "\alpha_c"] & Gc \ar[d, "Gf"]
\\ Gd \ar[r, "\alpha^{-1}_d", swap] \ar[rr, "1_{Gd}", bend right = 40, swap] & Fd \ar[r, "\alpha_d", swap] & Gd
\end{tikzcd}
\end{center}

\item What is a natural transformation between a parallel pair of functors between groups, regarded as one-object categories? 

Let $\ant{F}{G}$ be a natural transformation between functors $F, G: \bg \to \bgg$. The data of $\alpha$ consists of components $\alpha_g$ for each $g \in G$ such that $Fg \to Gg$. From the previous exercises, we know that functors $F, G$ of these categories translate into group homomorphisms $G \to G'$ mapping unital elements to unital elements, and preserving the group structure. There is precisely one component of $\alpha$:  $\alpha_*$. The picture being revealed is that of a composition between permutation of the group $\bg$ - an inner automorphism - and a homomorphism of groups. 

\item If you're into it, go for it.

\item In the notation of Example 1.4.7, prove that distinct parallel morphisms $f,g : c \rightrightarrows d$ define distinct natural transformations 

\[
\begin{aligned}
f_*,g_* : \cc(-,c) \Rightarrow \cc(-, d) \qquad \text{ and } \qquad f^*, g^* : \cc(d, -) \Rightarrow \cc(c, -)
\end{aligned}
\]

Lets suppose that $f,g$ were distinct, but defined indistinct natural transformations. This yields a contradiction via the naturality condition. lets write this one out. 

\item Recall the construction of the comma category for any pair of functors $F : \dd \to \cc$ and $G : \ee \to \cc$ described in the previous section's exercises. From this data, construct a canonical natural transformation $\alpha : F dom \nt G cod$ between the functors that form the boundary of the square

\begin{center}
\begin{tikzcd}
F \downarrow G \ar[r, "cod"] \ar[d, "dom", swap] & \ee \ar[d, "G"]
\\ \dd \ar[r, "F", swap] \ar[ur, Rightarrow, "\alpha" , shorten = 5mm] & \cc
\end{tikzcd}
\end{center}

Note that both $F \cdot dom$ and $G \cdot cod$ are $\cc$-valued functors, and that $dom (d, e, f) = d, cod (d, e, f) = e$. Also note that by the construction of $dom$ and $cod$, $(F \cdot dom)(d,e,f) = Fd$ and $(G \cdot cod)(d,e,f) = Ge$. Hence, the naturality square for $\alpha$ reduces to the following data: 

\begin{center}
\begin{tikzcd}
Fd \ar[r, "f"] \ar[d, "h", swap] & Ge  \ar[d, "k"]
\\ Fd' \ar[r, "f'", swap] & Ge' 
\end{tikzcd}
\end{center}
\end{enumerate}

\subsection{Equivalence of Categories}

This section deals with notions of equivalence and different flavors of functor.

\begin{enumerate}[(i)]
	\item Fixing a parallel pair of functors $F,G: \cc \rightrightarrows \dd$, natural transformations
$\alpha : F \Rightarrow G$ correspond bijectively to functors $H : \cc \times \two \to \dd$ such that $H$ restricts along $i_0$
and $i_1$ to the functors $F$ and $G$, i.e., so that
	
\begin{center}
\begin{tikzcd}
	\cc \ar[r, "i_0"] \ar[dr, "F", swap] & \cc \times \two \ar[d, "H"] & \dd \ar[l, "i_1", swap] \ar[dl, "G"]
	\\ & \dd
	
\end{tikzcd}
\end{center}

We begin by defining $H \in \{ H : \cc \times \two \to \dd \}$ such that $H(c, 0) := F$ and $H(c, 1) := G$. Then, we note that, for the unique arrow $k : 0 \to 1 \in \two$, we can define $H(c, k) := \alpha_c : Fc \to Gc$ for all $\alpha \in \cat{Nat}(F, G)$ and every $c \in \cc$. This construction forms one half of the bijection $\cat{Nat}(F, G) \cong \{ H : \cc \times \two \to \dd \}$. Next, we define the other half of the bijection as follows: let $H \in \{ H : \cc \times \two \to \dd \}$. Note that $H$ immediately induces a natural transformation for for all pairs $(c, k)$ where $c \in \cc$ and $k$ is defined as above. We'll prove that $H$ is indeed natural for $f: c \to d$: 

\begin{center}
\begin{tikzcd}
	H(c, 0) := Fc \ar[r, "{H(c, k) := \alpha_c}"] \ar[d, "{H(f, 0) \cong Ff}", swap] & H(c, 1) := Gc \ar[d, "{H(f, 1) \cong Gf}"]
	\\ H(c', 0) := Fc' \ar[r, "{H(c', k) := \alpha_{c'}}", swap] & H(c', 1) := Gc'
	
\end{tikzcd}
\end{center}

\item Segal defined a category $\Gamma$ as follows: 

{\addtolength{\leftskip}{5mm}

$\Gamma$ is a category whose objects are all finite sets, and whose morphisms from $S$ to $T$ are maps $\theta : S \to P(T)$ such that $\theta(\alpha)$ and $\theta(\beta)$ are disjoint when $\alpha \neq \beta$. The composite $\theta : S \to P(T)$ and $\phi : T \to P(U)$ is $\psi : S \to P(U)$ where $\psi(\alpha) = \bigcup_{\beta \in \theta(\alpha)} \psi(\beta)$.
}

Prove that $\Gamma$ is equivalent to the opposite of the category $\cat{Fin_*}$ of finite pointed sets. In particular, functors introduced in Example 1.3.2(xi) define presheaves on $\Gamma$.

Let us show that $\Gamma$ is a category. First, we must show that identities and composition are well-defined, and afterwards, that this composition is associative. Note that composition is given to us by the definition above. In particular, the identity morphism for any $S$ is $1_S : S \to P(S) $. For any other $f : S \to P(T)$, we have that $f1_S(S) = f(S) = \bigcup_{\beta \in 1_S(S) = S} 1_S(\beta)$. Note that $1_S(\alpha)$ for $\alpha \in P(S)$ restricts to $1_{\alpha}$. Hence, $\bigcup_{\beta \in S} 1_S(\beta)$ amounts to $\bigcup_{\beta \in S} 1_\beta(\beta)$, and hence, is simply $S$. Thus, $f1_S = f$, Likewise, the same argument can be made for $1_Tf$. 

Let $f : S \to P(T), g: T \to P(U)$, and $h: U \to P(W)$. We will show that the composite $hgf$ is associative: 

\begin{equation*}
	\begin{aligned}
		(h(gf))(\alpha) & = \bigcup_{\gamma \in gf(\alpha)} h(\gamma) \\
				      & = \bigcup_{\gamma \in \bigcup_{\beta \in f(\alpha)} g(\beta)} h(\gamma)\\
				      & = \bigcup_{\beta \in f(\alpha)} hg(\beta)  \\
				      &= ((hg)f)(\alpha)
	\end{aligned}
\end{equation*} 

The latter 2 steps are achieved by noting that the composition $hg(\beta) :=  \bigcup_{\gamma \in g(\beta)} h(\gamma)$. Now let's use these facts to prove that there is an equivalence $\Gamma  \simeq \cat{Fin_*}^{op}$. Define the functor $F : \Gamma \to \cat{Fin_*}^{op}$ as follows: 

Define $F$ such that for any morphism in $\Gamma$, $f : S \to T$, we have a map in $\cat{Fin_*^{op}}: Ff : T \times \{\beta\} \to S \times \{\alpha\}$ mapping each $\beta \in P(T)$ to its preimage $\alpha \in S$. Check that this composes, and that the identity for such a map is $1_S : S \times \{S\} \to S \times \{S\} \in \cat{Fin_*^{op}}$ - the image of $1_S \in \Gamma$. Note that this is a contravariant mapping, since $F(gf) = Ff \cdot Fg : U \times \{\gamma\} \to S \times \{\alpha\}$. Hence $F$ is a functor.

Lets define the other half in meeting.


\item Prove Lemma 1.5.10: Any morphism $f : a \to b$ and fixed isomorphisms $a \cong a'$ and $b \cong b'$ determine a unique morphism $f' : a' \to b'$ so that any of, or equivalently, all of the for diagrams commute: 


\[
	\begin{aligned}
		\begin{tikzcd}
			 a \ar[d, "f", swap] & a' \ar[l, "\cong", swap] \ar[d, "f'"] \\
			b \ar[r, "\cong", swap] & b'
		\end{tikzcd}
		\qquad
		\begin{tikzcd}
			a \ar[d, "f", swap] \ar[r, "\cong"] & a' \ar[d, "f'"] \\
			b \ar[r, "\cong", swap] & b'
		\end{tikzcd}
		\qquad
		\begin{tikzcd}
			a \ar[d, "f", swap] & a' \ar[l, "\cong", swap] \ar[d, "f'"] \\
			b & b' \ar[l, "\cong"]
		\end{tikzcd}
		\qquad
		\begin{tikzcd}
			a \ar[d, "f", swap] \ar[r, "\cong"] & a' \ar[d, "f'"] \\
			b  & b' \ar[l, "\cong"]
		\end{tikzcd}
	\end{aligned}
\]

We'll prove one of these, and the rest will follow. Let's do the first one. Let $a \cong a', b \cong b'$, and suppose we have $f : a \to b$. The directional arrows of the isomorphism are a red-herring: in fact, an isomorphism goes both ways! Hence, we can define $f' : a' \to b'$: Denote the top and bottom isomorphisms $k, k'$ for legibility, and define $f'$ to be the composite $f' = k'fk$. Let's prove that this is a unique morphism. Suppose that there were another $g : a' \to b'$ determined by $f, k,$ and $k'$. Then, we can show this yields an equality mechanically.

\item Show that full and faithful functor $F: \cc \to \dd$ both \textbf{reflects} and \textbf{creates isomorphisms}. That is, show: 
	\begin{enumerate}[i]
		\item If $f$ is a morphism in $\cc$, so that $Ff$ is an isomorphism in $\dd$, then $f$ is an isomorphism. 
		\item If $x$ and $y$ are objects in $\cc$ so that $Fx$ and $Fy$ are isomorphic in $\dd$, then $x$ and $y$ are isomorphic in $\cc$.
	\end{enumerate}
	The converses of these statements hold for any functor. 
	
Lets do $i$ first. Suppose $f : c \to d$ in $\cc$ such that $Ff$ is an isomorphism in $\dd$ with inverse $k$. Since the map $\cc(c, d) \to \dd(Fc, Fd)$ is a bijection, we know that there is some $g \in \cc(d, c)$ such that $Fg = k$. Composing with $f$, we have that $F(fg) = Ff \cdot Fg = Ff \cdot k = 1_{Fc}$. Hence, $fg$ must be an isomorphism since identities map to identities uniquely. Rinse and repeat for $F(gf)$. 

Now lets do $ii$. Suppose $x, y \in \cc$ such that $Fx \cong Fy \in \dd$. Note that this means there is an isomorphism $k \in \dd(Fx, Fy)$. Since the map $\cc(x, y) \to \dd(Fx, Fy)$ is bijection, we must have that $k$ corresponds with a map $k' \in \cc(x, y)$, and since $F$ is full, we have that such a $k'$ may yield $Fk' = k$. By $i$, this implies that $k'$ is an isomorphism, whence $x \cong y$. 

\item Show an example of a faithful functor that does not reflect isomorphisms. 

Example: Consider two topological spaces which are not homeomorphic, with the same number of elements (say, the Mobius band and $\mathbb{R}$). Consider the forgetful functor $U : \cat{Top} \to \cat{Set}$. As sets, these two are isomorphic (they share cardinalities), however, as topological spaces, they are not isomorphic and never will be. Hence, $U$ does not reflect isomorphisms!
\end{enumerate}

\end{document}  